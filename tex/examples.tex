% I use the eplain pkg. Helps with cross referencing.
\input eplain
% To enable hyperlinks we need this line
\enablehyperlinks

%======================================================
% TOPIC: Centering text
%======================================================
\centerline{This text is centered}

%======================================================
% TOPIC: Inserting an image:
%======================================================
% Look into texdoc pdftex and look for the PDFTEX primitives->graphics
Here is an image:

\pdfximage width .15 \hsize{images/sample.png}
\pdfrefximage\pdflastximage

Here is image the image, but a little larger:

\pdfximage width .25 \hsize{images/sample.png}
\pdfrefximage\pdflastximage

%======================================================
% TOPIC: Items
%======================================================
\item{1.} This is an item
\itemitem{1.1} This is a sub item

%======================================================
% TOPIC: Breaks, Page Breaks
%======================================================
% This command forces a page break
\vfil\break

%======================================================
% TOPIC: Equations
%======================================================
% Fractions
$$ 1 \over N \eqdef{Fractions}$$

% Conditional Equations
% See page 175 in The Texbook
$$ x =\cases{x,&if $x=0$;\cr -x,&otherwise.\cr}\eqdef{Conditional}$$

% Large left brace, putting the dot at the end of the \right makes it output nothing
$$ x = \left\lbrace _{y}^{z \over 4}\right.\eqdef{leftRight}$$

% Normal Text in equation
$$ x = \rm{max}(x, y)$$

% Aligning equations
$$
\eqalign{
L_i&=\sum_{j\neq y_i}\cases{
0,&if $s_{y_i}\ge s_j+1$\cr
s_j-s_{y_i}+1&otherwise\cr}\cr
&=\sum_{j\neq y_i}max(0,s_j-s_{y_i}+1)\cr
} %eqalign
\eqdef{Lec3:SVMLoss}
$$


%======================================================
% TOPIC: Tables
%======================================================
$$
\vbox{
\halign{#\hfil & \quad#\hfil & \quad#\hfil\cr
\noalign{\medskip\hrule\medskip}
\bf Command & \hfil \bf Description & \hfil \bf Example \cr
\noalign{\medskip\hrule\medskip}
\verbatim git add|endverbatim & add local changes to git's internal list of local changes & \verbatim git add <file>|endverbatim\cr
\verbatim git commit|endverbatim & commit you local changes (think, making a new record)  & \verbatim git commit -a -m ``message''|endverbatim \cr
\verbatim git push|endverbatim & push your changes upstream & \verbatim git push origin master|endverbatim\cr
\verbatim git log|endverbatim & view previous commits & \verbatim git log|endverbatim\cr
\verbatim git show|endverbatim & show contents of a commit. Default is last commit. & \verbatim git show|endverbatim\cr
} %halign
} %vbox
$$


%======================================================
% TOPIC: Eplain usage:
%======================================================
%======================================================
% TOPIC: Cross reference
%======================================================
\def\sectionword{Section}
\definexref{Section:}{Cross reference to nothing really}{section}
Now to reference this \ref{Section:}

%======================================================
% TOPIC: Code snippets
%======================================================
Here is a nice way to put code snippets, this uses the verbatim command from eplain
\medskip
\verbatim $ ls greektext-westcott-hort/parsed/
1CO.UWH  1TH.UWH  2JO.UWH  2TI.UWH  COL.UWH  HEB.UWH  JUDE.UWH  MT.UWH   RE.UWH
1JO.UWH  1TI.UWH  2PE.UWH  3JO.UWH  EPH.UWH  JAS.UWH  LU.UWH    PHM.UWH  RO.UWH
1PE.UWH  2CO.UWH  2TH.UWH  AC.UWH   GA.UWH   JOH.UWH  MR.UWH    PHP.UWH  TIT.UWH
|endverbatim
\medskip

%======================================================
% TOPIC: Hyperlink reference
%======================================================
\href{https://en.wikipedia.org/wiki/Computer_terminal}{terminal}


% Always need this at the end of a file
\bye
